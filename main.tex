%% If you want to use \orcid or the
%% academicons icons, add "academicons"
%% to the \documentclass options.
%% Then compile with XeLaTeX or LuaLaTeX.
% \documentclass[10pt,a4paper,academicons]{altacv}

%% Use the "normalphoto" option if you want a normal photo instead of cropped to a circle
% \documentclass[10pt,a4paper,normalphoto]{altacv}

\documentclass[10pt,a4paper]{altacv}

%% AltaCV uses the fontawesome and academicon fonts
%% and packages.
%% See texdoc.net/pkg/fontawecome and http://texdoc.net/pkg/academicons for full list of symbols.
%% When using the "academicons" option,
%% Compile with LuaLaTeX for best results. If you
%% want to use XeLaTeX, you may need to install
%% Academicons.ttf in your operating system's font %% folder.


% Change the page layout if you need to
\geometry{left=1cm,right=9cm,marginparwidth=6.8cm,marginparsep=1.2cm,top=1cm,bottom=1cm}

% Change the font if you want to.

% If using pdflatex:
\usepackage[utf8]{inputenc}
\usepackage[T1]{fontenc}
\usepackage[default]{lato}

% If using xelatex or lualatex:
% \setmainfont{Lato}

% Change the colours if you want to
\definecolor{VividPurple}{HTML}{008080}
\definecolor{SlateGrey}{HTML}{2E2E2E}
\definecolor{LightGrey}{HTML}{666666}
\colorlet{heading}{VividPurple}
\colorlet{accent}{VividPurple}
\colorlet{emphasis}{SlateGrey}
\colorlet{body}{LightGrey}

% Change the bullets for itemize and rating marker
% for \cvskill if you want to
\renewcommand{\itemmarker}{{\small\textbullet}}
\renewcommand{\ratingmarker}{\faCircle}


\begin{document}
\name{Ciaran McKey}
\tagline{}

\personalinfo{%
  \email{ciaranmckey@gmail.com}
  \phone{+44 7472 557 891}
  \location{Cambridge, UK}
  \linkedin{linkedin.com/in/ciaran-mckey}
  \github{github.com/thewatergategroups} 
}

\begin{fullwidth}
\makecvheader
\end{fullwidth}


%% Provide the file name containing the sidebar contents as an optional parameter to \cvsection.
%% You can always just use \marginpar{...} if you do
%% not need to align the top of the contents to any
%% \cvsection title in the "main" bar.
\cvsection[page1sidebar]{Experience}
%talk about ARGO CD provisioning, authing, microservice resilience, database migration, 

\cvevent{Engineering Lead}{ \includegraphics[scale=0.025]{dt_logo.jpeg} Darktrace | Cyber Security | 3000 Employees | Global }{Jan 2024 - Present}{Cambridge, UK}
\textbf{Technologies}

{\faExclamation}{   \textbf{All technologies from the previous role still apply.}}

\cvtag{Golang}
\cvtag{Rust}
\cvtag{ReactJS}
\cvtag{Figma}

\divider
\textbf{Overview}

I have been given a team to lead following the successful launch of the Cloud Security product. My team goals are to further Research and Development within the Cloud Security space outside of the core product offering. This means identifying capabilities the product is lacking, understanding the requirements, designing and building these features to improve the product offering. This includes working with stakeholders across departments. Product to identify missing features. UX to design workflows for these features. UI to integrate with the backend and DevOps to get required pipelines and infrastructure in place.   
\newline
\newline
\textbf{Responsibilities}
\begin{itemize}
  \item Working closely with UX to standardise workflows and improve useability in the existing product with an aim to bring more useful insights and finding to the user in a more digestable and consistent way.
  \item Working closely with UI to build data frameworks around UI components in order to standardise how data is served for quick itteration in the future. 
  \item Working closely with engineering leads to standardise processes and functionality across the product suite to create a more seamless experience.
  \item Architect, build and deploy all facets of production applications on Kubernetes to serve new product services.
  \item Manage priorities, Assign tasks and deadlines to team members and all other clarical work associated with management
\end{itemize}
\divider

\cvevent{Software Engineer}{ \includegraphics[scale=0.025]{dt_logo.jpeg} Darktrace | Cyber Security | 3000 Employees | Global }{Apr 2021 - Jan 2024}{Cambridge, UK}

\divider
\textbf{Technologies}

\cvtag{Python}
\cvtag{Asyncio}
\cvtag{Concurrent Futures}
\cvtag{PyTest}
\cvtag{Uvicorn}
\cvtag{FastAPI}
\cvtag{SQLAlchemy}
\cvtag{Alembic}
\cvtag{Celery}
\cvtag{Pydantic}
\cvtag{PostgreSql}
\cvtag{Redis}
\cvtag{Docker}
\cvtag{Kubernetes}
\cvtag{GitLab CI/CD}
\cvtag{Git}
\cvtag{AWS DynamoDb}
\cvtag{AWS RDS} 
\cvtag{AWS EC2}
\cvtag{AWS EKS}
\cvtag{AWS ECS}
\cvtag{AWS SQS} 
\cvtag{AWS IAM}
\cvtag{AWS WAF}
\cvtag{AWS S3}
\cvtag{AWS CloudTrail}
\cvtag{AWS Cloudformation}
\cvtag{AWS Lambda}
\cvtag{AWS API Gateway}
\cvtag{AWS Parameter Store}
\cvtag{AWS Secrets Manager}
\divider

\cvsection[page2sidebar]{Experience Continued}

\textbf{Overview}

Started on existing offerings within the SaaS space. Wrote the first line and much of the application code on the core Cloud Security product offering for AWS. Took it to market with a team that grew to 11 engineers while managing the production cycles, production debugging, client calls and supporting teams across DevOps, Support, Product Development and Internal Development to ensure smooth operation during release.
\newline
\newline
Moved from working on a monolithic based architecture to a containerised microservice architecture. In doing so we built standards within Darktrace for further microservice development. Ranging from fault tolerance, scaling, resilience, security, developmeent CI/CD and customer provisioning and deployment pipelines.

\divider
\textbf{Responsibilities}

\begin{itemize}
\item Architect, build and deploy microservices on Kubernetes to serve the backend of the product offering.
\item Build authentication mechanisms to ensure customer data separation in production environments.
\item Build customer provisioning pipelines.
\item Help solidify company microservice development best practices within Darktrace.
\item Manage production release cycles.
\item Talk with customers to help deploy the product and debug issues.
\end{itemize}
\divider
  
\cvevent{System and Backend Developer}{ \includegraphics[scale=0.025]{pwfl_logo.jpg} Play Well for Life | EduTech | 10 Employees | Regional }{Jan 2020 - Apr 2021}{Surrey, UK}
\textbf{Technologies}

\cvtag{AWS ECS}
\cvtag{AWS EC2}
\cvtag{Java}
\cvtag{Python}
\cvtag{SQL}
\cvtag{Bubble.io}
\cvtag{GitLab}
\cvtag{Linux}
\divider

\textbf{Overview}

Quick itteration development building MVP's for product ideas to take to investors and test on potential customers. Individual responsibilities included designing system architecture, building CI/CD pipelines, building provisioning pipelines, writing application code, deploying infrastructure and ensuring uptime. 
\divider

\textbf{Responsibilities}
\begin{itemize}
  \item Developed and deployed containerised microservices on AWS ECS
  \item Developed Java based microservice with an SQL datastore
  \item Worked with microcontrollers to develop hyrid system
  \item Built CI/CD pipelines
  \item Built Customer Portal with Bubble.io
\end{itemize}

\cvsection{Hobbies}

\cvachievement{\faMapMarker}{Rock Climbing}{}
\cvachievement{\faPlane}{Traveling}{}
\cvachievement{\faCode}{Coding}{}

\clearpage

\end{document}